\chapter{Préparation / Mise en Œuvre} %Trouver un titre plus accrocheur, plus pertinant
Le projet DIAMS a nécessité le déploiement et l'utilisation de nombreux outils et méthodes : aide à 
la gestion,
tests, analyse de code, gestionnaire de version, communications et conventions. Certaines 
fonctionnalités de DIAMS sont basées sur des bibliothèques externes ou sur d'autres applications. 
Nous détaillons ici le déroulement de ce qui constitue l'étape préliminaire du projet.

\section{Suivi de projet}
Le projet s'organise autour de la méthode Scrum. Pour cette méthode agile, nous utilisons l'outil
Youkan. Nous voulions utiliser l'outil Jazz d'IBM, qui offre des fonctionnalités 
de communication et de management très étoffées, mais nous avons dû renoncer car l'ensemble logiciel était payant.

%%Youkan, comment on le rempli ensembles, comment on valide avec le client, comment on vote les coûts.
\subsection{Youkan} %Screenshots please
Youkan est une application web qui permet de s'organiser autour de User Stories,
les différentes fonctionnalités d'un logiciel, qui sont rédigées par l'équipe.
Ces User Stories sont découpées en tâches. Chaque tâche est pondéré par un nombre de points choisi
lors d'un vote. Youkan est accessible par le client, lui offrant la possibilité de suivre l'avancement du
projet. La répartition du travail est basé sur le volontariat. Chaque semaine, nous organisons entre une et deux
réunions afin de faire le point (difficultés rencontrées, planification de risques, redistribution des tâches etc.),
et une fois par semaine, une réunion avec le client a lieu pour affiner la liste des User stories, discuter de
l'avancement, lister les problèmes rencontrés et l'impliquer dans le projet.\\
Afin d'éviter tous problèmes d'incompatibilité entre les outils utilisés, nous avons choisi un environnement de développement commun à tous.

%Listes de diffusion mail, convention [PCP] pour filtrer... faut vraiment mettre ça ? --> Non j'ai écrit ce
%commentaire pour le fun...
%%Réunions scrum, dire comme on en fait souvent pour faire le point. Comment on gère les risques (anticipation, priorité, contournement/solutions possibles). 
\subsection{Organisation de la communication}


\section{Environnement de développement}
\subsection{IDE}
Eclipse est la référence pour le développement sous Android, le système d'exploitation visé. De plus il est multi-plateforme et résout le problème de l'utilisation de différents systèmes d'exploitation par les membres de l'équipe de développement.
Son système de plugins permet de gérer tout le kit de développement Android. Nous utilisons également le DDMS (Dalvik Debug Monitor Server), qui permet de détecter et de résoudre des bugs tout en interagissant avec le système. 

\subsection{Style de codage}
Une convention de codage a été mise en place pour assurer une plus grande cohérence au sein de l'équipe, cette convention est partie intégrante de la documentation développeur.

\subsection{matériel de déploiement}
Nous utilisons une tablette tactile pour le déploiement de notre application. Cette tablette possède la version Android 2.1(API 7), qui n'est pas la plus récente. Par conséquent, certains services récents ne sont pas accessibles. 
%le fichier de convention je le cite dans la biblio ? --> Non.

\section{Outils de versionnage}
Le code est versionné sous git, pour que chacun dispose de son propre serveur. Cela procure l'avantage de réduire les \textbf{risques de conflits}. Le serveur web que nous avons choisi 
d'utiliser est GitHub. Il nous propose un traqueur de bogues, un système de commentaires du code pour un meilleur suivi du projet et un outil de visualisation de l'activité du projet sous forme de graphes d'arborescences. En sur-couche graphique de git, nous utilisons le logiciel Smartgit. Il propose des outils de comparaison de code et des
journaux visuels.
Nous versionnons le code en utilisant la méthode \textbf{PULL then PUSH} : seul le \textbf{Scrum Master} possède le droit d'intégrer des
sources sur la branche master (branche principale). Les autres membres envoient leurs versions sur d'autres
branches, mais seul le scrum master peut réaliser des fusions entre d'autres branches et la branche master. Les
autres développeurs peuvent demander une intégration de leurs modifications sur la branche principale. Nous utilisons ce système de branches pour séparer ce qui est en développement de ce qui est en phase d'intégration. L'arborescence représente aussi la répartition des fonctionnalités.
Afin de faciliter la mise en place et le travail nous avons décidé d'utiliser des outils d'automatisation.

\section{Serveurs d'automatisations}

\subsection{Maven : déploiement et dépendances}
Maven est un outil puissant permettant d'automatiser la mise en place d'un projet Java. Il gère la gestion des dépendances et des librairies utilisées. Cependant suite à des problèmes d'incompatibilité entre la version d'Android utilisée et le système de templates de Maven, nous avons dû renoncer à l'utiliser.
L'alternative utilisée est l'outil Ant et le générateur d'APK de Google (Application Package File) permettant d'installer des logiciels sous Android. Les dépendances sont gérées manuellement, ce qui ne pose pas de problème quand le nombre de dépendances est faible.

\subsection{Jenkins : tests et compilation}
Jenkins est un outil permettant via un serveur de réaliser automatiquement des tests. A chaque archivage du code, Jenkins peut exécuter les tests pour lesquels il est configuré et génère
un compte-rendu ainsi que des journaux détaillés. La prise en main de Jenkins a nécessité du temps et son exécution
se fait en local. Nous avons considéré et abandonné l'alternative proposée au CREMI, où le serveur est configuré pour fonctionner sur Savane, qui n'utilise pas git mais svn.

\section{Analyse de code}

Le projet est en Java, par conséquent le garbage collector gère la mémoire. Toutefois, étant donné que l'application doit 
fonctionner sur une tablette, il est nécessaire d'éviter de surcharger cette mémoire (ouvrir trop de fichiers en même temps par exemple). Les outils d'analyse de code aident à éviter les mauvaises pratiques qui génèrent des fuites mémoire.
\subsection{Sonar}
Sonar permet une analyse poussée du code (duplications de code, détection de bogues, couverture par des tests etc.) mais il est basé sur Maven. Par conséquent nous ne pouvons pas l'utiliser. Ainsi nous nous sommes orientés vers un autre outil.

\subsection{Lint}
Lint permet la détection de certaines erreurs de conception types et précise si des éléments sont
obsolètes par rapport à l'API visé, le but étant d'améliorer la lisibilité, la sécurité, la performance du code...

\section{Bibliothèques}
Pour ce projet nous nous sommes intéressé à des librairies offrant un grand nombre de fonctionnalités dont nous avions besoin : parser d'images et d'informations, gestionnaire de fichiers et traitement d'images. Deux libraires ont retenu notre attention.
\begin{itemize}
\item[\textbf{dcm4che:}]  est un projet open source en Java sur la médecine notamment la gestion des DICOM. Cependant il utilise le package Swing qui est incompatible avec Android.
\item[\textbf{pixelmed:}] projet open source offrant la gestion de fichiers sous la norme DICOM et des fonctionnalités de traitement d'images. Il utilise le package awt qui n'est pas implémenté sous 
Android.
\end{itemize}