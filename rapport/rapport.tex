\documentclass[a4paper, oneside, 10pt]{book}

\usepackage{fancyhdr,verbatim, listings, color} 
%\usepackage{shorttoc}
%\usepackage[french]{minitoc}
\usepackage[pdftex]{graphicx} 
\usepackage[utf8]{inputenc}
\usepackage[T1]{fontenc}
\usepackage[frenchb]{babel}
\usepackage{amsmath,amssymb,mathrsfs}
\usepackage{url}

\definecolor{colKeys}{rgb}{0,0,1} 
\definecolor{colIdentifier}{rgb}{0,0,0} 
\definecolor{colComments}{rgb}{0,0.5,1} 
\definecolor{colString}{rgb}{0.6,0.1,0.1} 

\usepackage[pdftex,
    bookmarks         = true,
    bookmarksnumbered = true,
    bookmarks = true, 
    pdfstartview      = FitH,
    bookmarksopen = true, 
    bookmarksopenlevel = 1, 
    colorlinks        = false,
    urlcolor          = red,
    pdfborder         = {0 0 0}
    ]{hyperref}
\hypersetup{
    pdfauthor   = {Clément Badiola, Alexandre Perrot, Samuel Da Silva, Reda Lyazidi},% 
    pdftitle    = {Conduite_projet},%
    pdfsubject  = {Réalisation d'une IHM sur tablette tactile},% 
    pdfkeywords = {},% 
    pdfcreator  = {PDFLaTeX},% 
    pdfproducer = {PDFLaTeX}} 

\lstset{
float=hbp,
basicstyle=\ttfamily\small,
identifierstyle=\color{colIdentifier},
keywordstyle=\color{colKeys},
stringstyle=\color{colString},
commentstyle=\color{colComments},
%language=c,
columns=flexible,
tabsize=2,
frame=trBL,
frameround=tttt,
extendedchars=true,
showspaces=false,
showstringspaces=false,
numbers=left,
numberstyle=\tiny,
breaklines=true,
breakautoindent=true,
captionpos=b,
commentstyle=\textit
}


\author{Clément~\textsc{Badiola} \and Alexandre~\textsc{Perrot} \and Samuel~\textsc{Da~Silva} \and Reda~\textsc{Lyazidi}  \and \\
  Client : Hugo~\textsc{Balacey} \\
  }
\title{Réalisation d'une IHM sur tablette tactile}
\date{\today}

%% Define a new 'leo' style for the package that will use a smaller font.
\makeatletter
\def\url@leostyle{%
  \@ifundefined{selectfont}{\def\UrlFont{\sf}}{\def\UrlFont{\small\ttfamily}}}
\makeatother
%% Now actually use the newly defined style.
\urlstyle{leo}

\pagestyle{fancy} 

\begin{document}

\fancyhf{} 
\renewcommand{\chaptermark}[1]{\markboth{#1}{}} 
\renewcommand{\sectionmark}[1]{\markright{\thesection\ #1}} 
\fancyhead[R]{\bfseries\thepage}% Left Even, Right Odd - No de page 
\fancyhead[L]{\bfseries\rightmark} % Left Odd - Titre section 
\renewcommand{\headrulewidth}{0.5pt}% filet en haut de page 
\addtolength{\headheight}{0.5pt} % espace pour le filet 
\renewcommand{\footrulewidth}{0pt} % pas de filet en bas 
\fancypagestyle{plain}{ % pages de tetes de chapitre 
\fancyhead{} % supprime l’entete 
\renewcommand{\headrulewidth}{0pt} % et le filet 
}

\fancypagestyle{plain}{
	\fancyhf{}
	\fancyfoot[C]{\thepage}
	\renewcommand{\headrulewidth}{0pt}
	\renewcommand{\footrulewidth}{0pt}}

%\dominitoc
%\dopartoc

\maketitle

\frontmatter
\tableofcontents

\mainmatter

\chapter{Prépation/Mise en Oeuvre}

\section{Serveurs}

- Automatisation

\section{Analyses}

- Code
- Fuite mémoire

\section{Bibliothèque}

\chapter{Architecture/Technique}

Blablabla

\section{Dicom}

- Système de visualisation (Houndsfield)

\section{Moteur de réseau}

\section{Gestionnaire de mémoire}

\chapter{Prépation/Mise en Oeuvre}

\section{Tests}

\subsection{Sous section 1}

\subsection{Sous section 2}

Liste d'item :
\begin{itemize}
\item Item 1
\item Item 2
\item Item 3
\end{itemize}



\appendix

\backmatter
\nocite{*}
\bibliographystyle{plain-fr}
\bibliographystyle{unsrt}
\bibliography{bibliographie}

\end{document}
